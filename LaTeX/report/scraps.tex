% \begin{tikzpicture}
% \begin{scope}[scale=.55]
%   \draw (0,0) ellipse (.625 and 0.25) node[midway,below](link1){};
%   \draw (-.625,0) -- (-.625,-1.5);
%   \draw (-.625,-1.5) arc (180:360:.625 and 0.25);
%   \draw (.625,-1.5) -- (.625,0);
% \end{scope}
% \begin{scope}[yshift = 2cm, scale=.55,rotate=90]
%   \draw (0,0) ellipse (.625 and 0.25);
%   \draw (-.625,0) -- (-.625,-1.5);
%   \draw (-.625,-1.5) arc (180:360:.625 and 0.25) node[midway,above](link2b){};
%   \draw (.625,-1.5) -- (.625,0);
% \end{scope}
% \draw (link1) -- (link2b);
% \end{tikzpicture}

% \begin{tikzpicture}
% \draw[->] (0,0,0) -- ++(1,0,0) node()[right]{$x$};
% \draw[->] (0,0,0) -- ++(0,1,0) node()[right]{$y$};
% \draw[->] (0,0,0) -- ++(0,0,1) node()[right]{$z$};
% \end{tikzpicture}


% The forward kinematics of the manipulator can be found using the Denavit–Hartenberg convention. \\
% \begin{table}[htp]
%   \center
%   \caption{DH Table for 6 DOF Manipulator}
%   \label{table:dh}
%   \begin{tabular}{C{1cm}|C{1.5cm}|C{2cm}|C{1cm}|C{1cm}}
%     DH & $d_i$ & $\theta_i$ & $a_i$ & $\alpha_i$ \\ \hline
%     1 & $\ell_1$ & $\theta_1$ & 0 & $\sfrac{\pi}{2}$ \\
%     2 & $-d$ & $\theta_2$ & $\ell_2$ & 0 \\
%     3 & 0 & $\theta_3 + \sfrac{\pi}{2}$ & 0 & $\sfrac{\pi}{2}$ \\
%     4 & $\ell_3 + \ell_4$ & $\theta_4$ & 0 & $-\sfrac{\pi}{2}$ \\
%     5 & 0 & $\theta_5$ & 0 & $\sfrac{\pi}{2}$ \\
%     6 & $\ell_6$ & $\theta_6$ & 0 & 0 \\
%   \end{tabular}
% \end{table}

% \begin{equation}
% A = Rot_{z,\theta}~Trans_{z,d}~Trans_{x,a}~Rot_{x,\alpha}
% \label{eq:J1}
% \end{equation}%
% Given an arbitrary homogeneous matrix $T_i^{i-1}$ (computed by matrix multiplication of $A$ matrices $\rightarrow \big[A_1A_2\cdots A_{i-1}A_i\big]$), the orientation vector $\bar{z}_i$ (with respect to $\varphi$, $\theta$ and $\psi$) and the relative joint position (displacement) vector $\bar{o}_i$ (with respect to $x$, $y$, and $z$) can be obtained via the 3$^{rd}$ and 4$^{th}$ columns of the matrix respectively,
% as shown in Equation \ref{eq:J2} (given $\beta$ as an arbitrary rotation angle about the $z$-axis).
% \begin{equation}
%   T_i^{i-1} =
% \begin{bmatrix}
%   c_{\beta} & -s_{\beta} & z_i^{\varphi} & o_i^x \\
%   s_{\beta} & c_{\beta} & z_i^{\theta} & o_i^y \\
%   0 & 0 & z_i^{\psi} & o_i^z \\
%   0 & 0 & 0 & 1 \\
% \end{bmatrix}
% \label{eq:J2}
% \end{equation}
