\documentclass[12pt]{report}
\usepackage[margin=1in]{geometry}
\usepackage{setspace} % for single/doublespacing commands
\usepackage{graphicx} % including graphics
\usepackage{sectsty} % sexy section headings
% \usepackage{pdfpages} % including multipage pdfs
\usepackage[export]{adjustbox} % for graphic frames and center
\usepackage{siunitx}
% \usepackage[numbered]{matlab-prettifier} % including matlab w/ syntax highlighting
% \usepackage[T1]{fontenc} % prettier matlab font
% \usepackage{xfrac} % more legible inline fractions (\sfrac)
\usepackage{lmodern} % font package for above
% \usepackage{multicol} % multiple columns
\usepackage[justification=centering]{caption} % figure captions (force centering)
% \usepackage{amsmath} % more math symbols and shit
\usepackage{enumitem} % add arguments for enumerate to change style
\usepackage[list=true]{subcaption} % subfigures with list of figure support
\usepackage{multirow}
% \usepackage{mathtools}
\usepackage{booktabs}
\usepackage{color}
\usepackage{ulem}
% \usepackage{blindtext}
\usepackage[numbers]{natbib}
\usepackage{contour}
\usepackage{tabularx}
% \usepackage{circuitikz} % drawing fancy shit
% \usepackage{cancel} % arrow and cross math cancel symbol
% \usepackage{lineno}
\usepackage{framed}
\usepackage{amssymb} % special math symbols
\usepackage{listings}
\usepackage{array}
% \usepackage{BOONDOX-cal} % fancy mathtype script
\usepackage{fancyhdr}
% \usepackage{flowchart}
\usepackage{color, colortbl}
\usepackage{tocloft}
\usepackage{url}
\usepackage{etoolbox}

\setlength{\parskip}{\baselineskip}%
\setlength{\parindent}{0pt}%
\setcounter{secnumdepth}{5}
\renewcommand{\bibname}{References}
\sisetup{output-exponent-marker=\ensuremath{\mathrm{e}}}
\newcommand{\PreserveBackslash}[1]{\let\temp=\\#1\let\\=\temp}
\newcolumntype{C}[1]{>{\PreserveBackslash\centering}p{#1}}
\newcolumntype{R}[1]{>{\PreserveBackslash\raggedleft}p{#1}}
\newcolumntype{L}[1]{>{\PreserveBackslash\raggedright}p{#1}}
\lstMakeShortInline[style=Matlab-editor]| % matlab inline escape character
\graphicspath{{images/}}
\renewcommand\thesection{\arabic{section}}
\renewcommand\labelitemi{---}
\lstset{numberstyle=\ttfamily\small\color{gray}}
% \renewcommand\linenumberfont{\ttfamily\small\color{gray}}
% \setlength\linenumbersep{6mm}
% \hbadness=99999  % or any number >=10000
\apptocmd{\sloppy}{\hbadness 10000\relax}{}{}
% \usetikzlibrary{arrows,calc,patterns,angles,quotes}
% \usetikzlibrary{shapes.geometric}
% \usetikzlibrary{decorations.pathmorphing,decorations.pathreplacing} % for snakes!
% \usetikzlibrary{positioning, circuits.logic.US}
\setlength{\cftbeforetoctitleskip}{-2em}
% \newcommand{\Lag}{\mathcal{L}} % lagrangian L
\allsectionsfont{\raggedright}

\begin{document}
\normalem
\begin{titlepage}
\flushleft
\doublespacing
\Large
\textsc{Test Document} \\
\normalsize
Trey Dufrene, Zack Johnson, David Orcutt, Alan Wallingford, Ryan Warner
\vfill
\center
\includegraphics[width=.45\textwidth]{logo}
\vfill
\flushleft
ME 407 \\
Preliminary Design of Robotic Systems \\
Embry-Riddle Aeronautical University \\
\vspace{2ex}
\begin{minipage}[c]{.5\textwidth}
\flushleft
\includegraphics[width=.95\textwidth]{erau}
\end{minipage}%
\begin{minipage}[c]{.5\textwidth}
\flushright
\includegraphics[width=.8\textwidth]{text}
\end{minipage}
\end{titlepage}

\pagenumbering{roman}
% \begin{abstract}
  % Wordy words
% \end{abstract}
{\tableofcontents\let\clearpage\relax\listoffigures}
% {\tableofcontents\let\clearpage\relax\listoffigures\let\clearpage\relax\listoftables}
\clearpage
\newpage

% \section*{List Of Acronyms and Abbreviations}

% \begin{tabular}{rl}
%   $G$~:&Center of gravity of the bar \\
%   $\ell_0$~:& Spring unstretched length  \\
% \end{tabular}
% \normalsize
% \flushleft
% \singlespacing
% \newpage
\pagenumbering{arabic}

\section{Introduction}
\raggedright
\begin{figure}[htp]
  \centering
  \includegraphics[frame,width=.75\textwidth]{model}
  \caption{Overview of Physical System}
  \label{fig:model}
\end{figure}
\section{Design Requirements}
\subsection{Hardware}

\subsubsection{The system shall cost the end-user no more than \$1000.}
\begin{enumerate}[label=\thesubsubsection.\alph*,leftmargin=3cm,font=\itshape]
  \item \textit{The cost for the MEIOSIS team to develop the manipulator shall cost no more than \$800.}
\end{enumerate}

\subsubsection{The system shall be fully dexterous without being kinematically redundant.}
\begin{enumerate}[label=\thesubsubsection.\alph*,leftmargin=3cm,font=\itshape]
\item \textit{The system shall consist of six rotational joints connected by four links. The last three joints will create a spherical wrist.} \\
As defined \cite{robo}, “A manipulator having more than six DOF is referred to as a kinematically redundant manipulator (5).” A manipulator with less than six degrees of freedom will not be fully dexterous within it's workspace. \emph{Figure \ref{fig:zero}} (pg. \pageref{fig:zero}) shows a six degree-of-freedom rotary manipulator with it's coordinate frames in zeroed positions. \\
~\\
\item \textit{The system shall have no link offsets.} \\
Link offsets are commonly used to avoid singularities. However, having a link offset prevents the manipulator's dexterous workspace from being a complete hemispherical shell.
\end{enumerate}

\subsubsection{The system end effector shall maintain a positional accuracy magnitude of \(\pm 1\) mm and an orientation accuracy of \(\pm 5^{\circ}\) eigen angle from the base frame.}
To ensure that the robot has educational value, the accuracy must be defined so that any desired positions and movements are achieved.
\begin{enumerate}[label=\thesubsubsection.\alph*,leftmargin=3cm,font=\itshape]
  \item \textit{The system shall possess the ability to calibrate the end effector position and orientation to within 0.5 mm and 1 degree of the manipulator’s precision.}\\
The addition of a calibration process allows the removal of any systematic errors, such as drift. The theoretical limit of the calibration process is the difference between the precision and accuracy metrics of the system.
\end{enumerate}
\subsubsection{The system end effector shall maintain a pose repeatability magnitude between 0.1—1.5 mm for the position and \(\pm 4^{\circ}\) eigen angle from the base frame for the orientation.}
\begin{enumerate}[label=\thesubsubsection.\alph*,leftmargin=3cm,font=\itshape]
  \item \textit{Joint one and two of the system shall possess an angle error of no more than .025 degrees.} \\
  Being that joint one and two are the first two rotational elements in the system, their error will propagate the most to the end effector's position.
  \item \textit{Joint three of the system shall possess an angle error of no more than .03 degrees.} \\
  Since joint three is closer to the end effector it's error will not propagate as severely throughout the system.
  \item \textit{Joints four, five, and six shall possess an angle error of no more than .29 degrees.} \\
  The spherical wrist is the closest to the end effector's final position and therefore has the least error propagation.
\end{enumerate}
\subsubsection{The system’s reachable workspace shall be a hemisphere with a radius of 300-700 mm.}
This workspace will provide enough movement to manipulate objects in order to perform basic tasks.
\begin{enumerate}[label=\thesubsubsection.\alph*,leftmargin=3cm,font=\itshape]
  \item \textit{The length of link one, two, three, four, and the wrist shall be 220.8 mm, 250 mm, 200 mm, 80 mm, and 52.5 mm respectively.} \\
  This results in a total height of 220.8 mm with a total reach of 582.5 mm in the zeroed configuration as shown in the configuration represented in \emph{Figure \ref{fig:zero}}.
\end{enumerate}

\subsubsection{The system’s dexterous workspace shall contain a hemispherical shell within the reachable workspace with a thickness of 280 mm.}
This workspace will provide enough movement to manipulate objects in order to perform basic tasks. 280mm is slightly greater than the length of letter paper.
\begin{enumerate}[label=\thesubsubsection.\alph*,leftmargin=3cm,font=\itshape]
  \item \textit{With respect to the kinematic model shown in Figure \ref{fig:zero}, the rotational limit of joint one, two, three, four, five, and six shall be \(\pm180^{\circ}\), \(\pm9.7^{\circ}\) to \(177.5^{\circ}\), \(-150.6^{\circ}\) to \(-19.3^{\circ}\), \(\pm180^{\circ}\), \(-180^{\circ}\) to \(-1.6^{\circ}\), and \(\pm180^{\circ}\) respectively.} \\
  To be fully dexterous within our 280 mm dexterous workspace the manipulator must have the joint angles specified above. The joint limitations were calculated by iteratively verifying the orientation about every point within the quarter hemisphere cross section seen in \emph{Figure \ref{fig:dex}} (Appendix, pg. \pageref{sec:app}).
  \begin{figure}[htp]
    \centering
    \includegraphics[width=.75\textwidth]{zero}
    \caption[Kinematic Model Representing Zeroed Configuration]{Kinematic Model Representing Zeroed Configuration \cite{robo}}
    \label{fig:zero}
  \end{figure}
\end{enumerate}
\subsubsection{The system shall have a removable end effector capable of picking and placing a low-odor chisel tip Expo dry erase marker.}
This creates a robot capable of performing a variety of basic tasks, which enhances its educational value.
\begin{enumerate}[label=\thesubsubsection.\alph*,leftmargin=3cm,font=\itshape]
  \item \textit{The system shall use a parallel gripper that can close to 18mm.} \\
  The diameter of a low-odor chisel tip Expo dry erase marker is approximately 18 mm.
  \item \textit{The end effector shall attach to the manipulator using screws configured in a pattern that can accommodate a Dynamixel AX-12A servo.} \\
  It is expected that a majority of end effector styles will have to accommodate for a servo to facilitate actuation, therefore a pattern was chosen to standardize the mounting.
\end{enumerate}

\subsubsection{The system shall be able to write with a low-odor chisel tip Expo dry erase marker.}
\begin{enumerate}[label=\thesubsubsection.\alph*,leftmargin=3cm,font=\itshape]
  \item \textit{The end effector shall be capable of applying a gripping force of 0.03 Newtons as to prevent slipping while writing.} \\
  The coefficient of friction between the Expo marker and paper can be approximated and given the weight of an Expo marker the approximate grip strength of the end effector can be calculated.
\end{enumerate}

\subsection{Software}
\subsubsection{The system shall be open source.}
This will create an easily obtainable, low cost method of distributing the system’s source code, which may be modified for personal use.
\begin{enumerate}[label=\thesubsubsection.\alph*,leftmargin=3cm,font=\itshape]
  \item \textit{The software shall be hosted publicly on an online repository and maintain an MIT license for distribution.} \\
  This allows the end-user to freely download and modify the code without licensing. The MIT license disregards any legal obligation to code upkeep and documentation by the original author.
\end{enumerate}
\subsubsection{The system shall be capable of operating given only desired end effector cartesian coordinates specified with respect to the base frame.}
\begin{enumerate}[label=\thesubsubsection.\alph*,leftmargin=3cm,font=\itshape]
  \item \textit{The system shall have a user interface capable of accepting user input.}  \\
    The system software interface facilitates an untrained user to operate without the advanced knoledge of the system's kinematics.
  \item \textit{The system shall be capable of performing floating point arithmetic.} \\
  The solution for the inverse kinematics requires the ability to perform high level arithmetic with little error.
\end{enumerate}

\newpage
\appendix
\renewcommand\thesection{\Roman{section}}
\renewcommand\thesubsection{\roman{subsection}}
\section*{Appendix}\label{sec:app}
\begin{figure}[htp]
  \centering
  \includegraphics[frame,width=.75\textwidth]{dex}
  \caption{Cross Section of Dexterous Workspace Quadrant}
  \label{fig:dex}
\end{figure}

\newpage
\bibliographystyle{plainnat}
\bibliography{robo}


\end{document}
