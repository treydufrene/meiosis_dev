
Based on the results of our testing it is not possible to fully determine that the manipulator functions in accordance with the requirements defined. Of our requirements listed earlier in the document, requirements 1.3, 1.4, 1.8, and 2.2 cannot be determined to have been met. Requirements 1.3 and 1.4 each define the necessary precision and accuracy required for this manipulator to be of educational value. Since the tests that would determine whether or not the manipulator adheres to these requirements can only be hypothesized and not actually performed, specifications 1.3.a-1.4.c could not be determined and therefore it is unknown as to whether or not the manipulator functions according to requirements 1.3 and 1.4.

Additionally, specification 1.8.a, which defines grip strength of the end effector, could not be determined as the end effector had not been ordered prior to spring break, so whether the end effector would have functioned according to requirement 1.8 could not be determined. Adherence to requirement 2.2 also cannot be determined. Due to the shift out of the lab, there was a large uncertainty as to how the robot would be controlled since there is no longer a physical robot to be moved. This resulted in inconclusive data regarding specification 2.2.a, relating to the user interface and input vector. Despite adhering to specification 2.2.b, it cannot be stated that the manipulator functions according to requirement 2.2.

Overall, it is not possible to say whether or not the manipulator meets the requirements defined in the preliminary design semester, especially since the success of the manipulator largely depended on the accuracy and precision requirements (1.3 and 1.4).
\subsubsection*{Changes and Recommendation}
Given the chance to start the project from the beginning, the number one thing we would change would be using the smart servos. While the servos seem convenient due to their ability to daisy chain and their internal controller, the servos had many downsides. The servos did not have a precise enough resolution to be used without additional gearing in the first three joints, although the MX series servos were significantly more precise than the AX series servo. The cheaper versions of the smart servos each had other issues; the AX-12A had weak torque output and a sixty degree dead zone in which no useful data could be read. The MX-12W was able to read plus or minus 28 full revolutions with a penalty to the precision, but the torque was extremely poor and unusable for our purposes. The more expensive MX series servo, the MX-64T, provided usable torque and the same precision as the MX-12W, but were \$300 each, so while the MX-64T servos could be used for this manipulator, they were quite expensive, and gearing was still necessary for angle precision and torque requirements if the servos were to operate at one fifth their rated stall torque as recommended by dynamixel. Options we could have considered more are DC motors or stepper motors

Another change we would have made would have been to thoroughly verify that the components that were picked would function how we expected them to, as that would have saved a lot of time and prevented weeks of work becoming obsolete. The AX-12A servos are a perfect example, as our “back of the envelope” calculations for the torque required by the servos was not accurate at all and was not corrected for over almost a month, after which the solution was to add gearing to the output of the servo. It then took another week or two to realize that the AX-12A servos had the dead zone, and therefore could not be used with gearing. These issues and lost time could have been avoided with minimally more effort into the verifications of the components compared to the effort that was wasted before the issues were found.

A final change that might be made given the opportunity to start again would be to potentially define more realistic requirements. As an example, having a requirement that created a need for motor/gearbox combinations that created a .025 degree precision and good torque output given an \$800 budget might not have been reliably attainable. The MEIOSIS team theoretically achieved the necessary angle precision and torque necessary to meet our strict requirement, but to do so three \$300 smart servos that on their own would have surpassed our budget if we had to purchase them were utilized, as well as 3D printed harmonic gearboxes using flexible PLA were used, and the reliability/precision of of both the harmonic gearboxes and the full system were not able to be tested so it is not known if the design actually met the accuracy and precision requirements.

For any future teams looking to work on the MEIOSIS manipulator, one recommendation would be to use the Rapid Prototype Lab for as many parts as possible, as the variance in the precisions of the prints done by the RPL and the in house printer made a difference. The in house printer tended to be used more for quick prototyping and general verification while the final versions of the components were sent to the RPL. Another thing to consider is how to get each joint within the manipulator to take the same amount of time to get to the end position. If the joints don’t reach the end position at the same time, the motion of the end effector will not be accurate to the path that the manipulator is trying to follow. It may also be worth switching to DC motors, and while it would require a PID controller to be written and motor drivers to be added, that process might still be less of a hassle than trying to use the smart servos.
