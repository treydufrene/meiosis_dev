\begin{lstlisting}[frame=lines,language=Python, caption=meiosis\_kinematics.py]
import numpy as np
import math as m
from meiosis_utils import matrixOp as mf


r1 = np.matrix([[0.00000],[0.00000],[0.12275]])
r2 = np.matrix([[0.00000],[0.00000],[0.00000]])
r3 = np.matrix([[0.00000],[0.26000],[0.00000]])
r4 = np.matrix([[0.00000],[0.20000],[0.00000]])
r5 = np.matrix([[0.00000],[0.07000],[0.00000]])
r6 = np.matrix([[0.00000],[0.09600],[0.00000]])

def FK(gamma):
    T1 = mf.rotz(gamma[0])
    T2 = T1*mf.rotx(gamma[1])
    T3 = T2*mf.rotx(gamma[2])
    T4 = T3*mf.roty(gamma[3])
    T5 = T4*mf.rotx(gamma[4])
    T6 = T5*mf.roty(gamma[5])

    Ir1 = r1
    Ir2 = Ir1 + T1*r2
    Ir3 = Ir2 + T2*r3
    Ir4 = Ir3 + T3*r4
    Ir5 = Ir4 + T4*r5
    Ir6 = Ir5 + T5*r6

    return Ir6

def IK(pos,R):
    pi = m.pi
    L1 = r1[2]
    L2 = r3[1]
    L3 = r4[1] + r5[1]
    npos = pos - R*r6
    xc = npos[0]
    yc = npos[1]
    zc = npos[2]

    #Inverse Position
    t1 = m.atan2(xc,yc) - pi/2
    if t1 < -pi:
        t1 = t1 + 2*pi
    if t1 > pi:
        t1 = t1 - 2*pi
    D = (xc**2 + yc**2 + (zc - L1)**2 - L2**2 - L3**2)/(2*L2*L3)
    t3 = m.atan2(-m.sqrt(1 - D**2),D)
    t2 = m.atan2(zc - L1, m.sqrt(xc**2 + yc**2) - atan2(L3*m.sin(t3), L2 + L3*cos(t3)))

    #Inverse Orientation
    T3 = mf.rotz(t1)@mf.rotx(t2)@my.rotx(t3)
    T = T.I*R
    t6 = m.atan2(T[1,0],-T[1,2])
    t4 = m.atan2(T[0,1],T[2,1])
    while t4 > pi/2:
        t4 -= pi
    while t4 < -pi/2:
        t4 += pi

    if t6 < 0:
        t6 += 2*pi

    if (m.sin(t4) > -10e-6) and (sin(t4) < 10e-6):
        t5 = m.atan2(T[2,1]/m.cos(t4),T[1,1])
    else:
        t5 = m.atan2(T[0,1]/m.sin(t4),T[1,1])

    return np.matrix([[t1],[t2],[t3],[t4],[t5],[t6]])
\end{lstlisting}
\vspace{10ex}

\begin{lstlisting}[frame=lines,language=Python, caption=meiosis\_servo.py]
#!/usr/bin/env python
# -*- coding: utf-8 -*-

"""basic readable servo commands"""

from dynamixel_sdk import *

# Control table addresses
ADDR_MX_TORQUE_ENABLE      = 24
ADDR_MX_GOAL_POSITION      = 30
ADDR_MX_PRESENT_POSITION   = 36

# Protocol version
PROTOCOL_VERSION            = 1.0

# Default setting
numdxl                      = 1;
BAUDRATE                    = 115200
DEVICENAME                  = '/dev/ttyUSB0'

gearidx = [(39.0)*(128.0/45.0), (39.0)*(128.0/45.0)]
offset = [0,13700]

# Initialize PortHandler instance
portHandler = PortHandler(DEVICENAME)
# Initialize PacketHandler instance
packetHandler = PacketHandler(PROTOCOL_VERSION)

class Servo:
    def __init__(self):
        self.data = []
        #self.initialize()

    def initialize(self):
        # Open port
        if portHandler.openPort():
            print("Succeeded to open the port")
        else:
            print("Failed to open the port")
            quit()
        # Set port baudrate
        if portHandler.setBaudRate(BAUDRATE):
            print("Succeeded to change the baudrate")
        else:
            print("Failed to change the baudrate")
            quit()

        # Enable Dynamixel Torque
        for i in range(0, numdxl):
            dxl_comm_result, dxl_error = packetHandler.write1ByteTxRx(portHandler, i, ADDR_MX_TORQUE_ENABLE, 1)
            if dxl_comm_result != COMM_SUCCESS:
                print("%s" % packetHandler.getTxRxResult(dxl_comm_result))
            elif dxl_error != 0:
                print("%s" % packetHandler.getRxPacketError(dxl_error))
            else:
                print("Dynamixel ID " + str(i) + " Torque has been enabled.")

    def close(self):
        # Disable Dynamixel Torque
        for i in range(0,numdxl):
            dxl_comm_result, dxl_error = packetHandler.write1ByteTxRx(portHandler, i, ADDR_MX_TORQUE_ENABLE, 0)
            if dxl_comm_result != COMM_SUCCESS:
                print("%s" % packetHandler.getTxRxResult(dxl_comm_result))
            elif dxl_error != 0:
                print("%s" % packetHandler.getRxPacketError(dxl_error))
            else:
                print("Dynamixel ID " + str(i) + " Torque has been disabled.")
        # Close port
        portHandler.closePort()

    def getPos(self, ID):
        return packetHandler.read2ByteTxRx(portHandler, ID, ADDR_MX_PRESENT_POSITION)[0]

    def setPos(self, ID, pos):
        dxl_comm_result, dxl_error = packetHandler.write4ByteTxRx(portHandler, ID, ADDR_MX_GOAL_POSITION, pos)
        #print(dxl_comm_result, dxl_error)

    def setVel(self, ID, vel):
        dxl_comm_result, dxl_error = packetHandler.write2ByteTxRx(portHandler, ID, 32, vel)

    def moving(self, ID):
        return packetHandler.read1ByteTxRx(portHandler, ID, 46)[0]

    def anyMoving(self, IDLIST):
        statusidx = [0,0]
        for i in range(0,len(IDLIST)):
            statusidx[i] = self.moving(IDLIST[i])
        return any(statusidx)

    def setRes(self, ID, res):
        dxl_comm_result, dxl_error = packetHandler.write1ByteTxRx(portHandler, ID, 22, res)

    def readRes(self, ID):
        return packetHandler.read1ByteTxRx(portHandler, ID, 22)[0]

    def disableTorque(self, ID):
        dxl_comm_result, dxl_error = packetHandler.write1ByteTxRx(portHandler, ID, ADDR_MX_TORQUE_ENABLE, 0)

    def enableTorque(self, ID):
        dxl_comm_result, dxl_error = packetHandler.write1ByteTxRx(portHandler, ID, ADDR_MX_TORQUE_ENABLE, 1)

    def setMultiturn(self, ID, on):
        if(on):
            dxl_comm_result, dxl_error = packetHandler.write2ByteTxRx(portHandler, ID, 6, 4095)
            dxl_comm_result, dxl_error = packetHandler.write2ByteTxRx(portHandler, ID, 8, 4095)
        else:
            dxl_comm_result, dxl_error = packetHandler.write2ByteTxRx(portHandler, ID, 6, 0)
            dxl_comm_result, dxl_error = packetHandler.write2ByteTxRx(portHandler, ID, 8, 4095)

    def setJA(self, IDLIST, angle):
        if(type(IDLIST) == int):
            IDLIST = [IDLIST]
        if(type(angle) == int):
            angle = [angle]
        for i in range(0,len(IDLIST)):
            self.setPos(IDLIST[i], int(round(angle[i] * gearidx[i] + offset[i])))
            print(int(round(angle[i] * gearidx[i] + offset[i])))

    def setOffset(self, ID, amount):
        offset[ID] = amount
        dxl_comm_result, dxl_error = packetHandler.write2ByteTxRx(portHandler, ID, 20, amount)
        return offset[ID]
\end{lstlisting}
\vspace{10ex}

\begin{lstlisting}[frame=lines,language=Python, caption=meiosis\_utils.py]
#!/usr/bin/env python
# -*- coding: utf-8 -*-

import numpy as np
# import math as m
from math import *

class matrixOp:
    def rotx(theta):
        return np.matrix([[1,0,0],[0,cos(theta),-sin(theta)],[0,sin(theta),cos(theta)]])

    def roty(theta):
        return np.matrix([[cos(theta),0,sin(theta)],[0,1,0],[-sin(theta),0,cos(theta)]])

    def rotz(theta):
        return np.matrix([[cos(theta),-sin(theta),0],[sin(theta),cos(theta),0],[0,0,1]])

    def skew(r):
        return np.matrix([[0,-r[2],r[1]],[r[2],0,-r[0]],[r[1],r[0],0]])
\end{lstlisting}
\vspace{10ex}

\begin{lstlisting}[frame=lines,language=Python, caption=twolink.py]
#!/usr/bin/env python
# -*- coding: utf-8 -*-

"""messy two link manipulator testing"""

import meiosis_encoder as ME
import RPi.GPIO as GPIO
import meiosis_servo as MS
import time
from math import *

ser = MS.Servo()
ser.initialize()

#ser.setRes(0,4)
#ser.setRes(1,4)
#ser.setMultiturn(0, True)
#ser.setMultiturn(1, True)

IDs = [0,1]
xoff = 0
yoff = 0

def twoLinkIK(x,y):
    l1 = 265.0
    l2 = 165.0
    D = (x**2 + y**2 - l1**2 - l2**2)/(2*l1*l2)
    theta2 = atan2(sqrt(1.0 - D**2),D)
    theta1 = atan2(y,x) - atan2(l2*sin(theta2),l1+l2*cos(theta2))
    print(degrees(theta1),degrees(theta2))
    return [degrees(theta1),degrees(theta2)]

def zeroArm():
    print("Init:")
    print(ser.getPos(0),ser.getPos(1))
    print("servo 0 offset: ")
    print(ser.setOffset(0,ser.getPos(0)))
    print("servo 1 offset: ")
    print(ser.setOffset(1,ser.getPos(1)))
    print("New:")
    print(ser.getPos(0),ser.getPos(1))

def moveArm(gamma):
    #global IDs
    ser.setJA(IDs, gamma)
    while (ser.anyMoving(IDs)):
        pass
    return True

def pick(x,y):
    #global xoff, yoff
    moveArm(twoLinkIK(x-xoff,y-yoff))

def place(x,y):
    moveArm(twoLinkIK(x,y))

try:
    print("Entering Loop, press Ctrl-C to escape!")
    pos = ser.getPos(0)*ser.readRes(0)
    print(ser.getPos(0))
    try:
        dir = raw_input('Enter Direction (u/d), or (f) for Finished:')
        zeroing = True
        while zeroing:
            while dir == 'u':
                userInput = raw_input()
                if userInput == 'f':
                    zeroing = False
                    break
                elif userInput == 'd':
                    dir = 'd'
                else:
                    pos += 50
                    ser.setPos(0,pos)
                    print(ser.getPos(0))
            while dir == 'd':
                userInput = raw_input()
                if userInput == 'f':
                    zeroing = False
                    break
                elif userInput == 'u':
                    dir = 'u'
                else:
                    pos -= 50
                    ser.setPos(0,pos)
                    print(ser.getPos(0))
        print(ser.getPos(0))

    except KeyboardInterrupt:
        print("interrupted")


    '''
    print(ser.getPos(0))
    ser.setPos(0,10000)
    while ser.moving(0):
        pass
    print(ser.getPos(0))
    #pick(250,50)
    #place(-250,50)
'''
except KeyboardInterrupt:
    print("\nInterrupted!")

print("Disabling Servo Interface...")
ser.close()
\end{lstlisting}
